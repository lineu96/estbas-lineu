% Options for packages loaded elsewhere
\PassOptionsToPackage{unicode}{hyperref}
\PassOptionsToPackage{hyphens}{url}
%
\documentclass[
  ignorenonframetext,
  serif,
  professionalfont,
  usenames,
  dvipsnames,
  aspectratio = 169]{beamer}
\usepackage{pgfpages}
\setbeamertemplate{caption}[numbered]
\setbeamertemplate{caption label separator}{: }
\setbeamercolor{caption name}{fg=normal text.fg}
\beamertemplatenavigationsymbolsempty
% Prevent slide breaks in the middle of a paragraph
\widowpenalties 1 10000
\raggedbottom
\setbeamertemplate{part page}{
  \centering
  \begin{beamercolorbox}[sep=16pt,center]{part title}
    \usebeamerfont{part title}\insertpart\par
  \end{beamercolorbox}
}
\setbeamertemplate{section page}{
  \centering
  \begin{beamercolorbox}[sep=12pt,center]{part title}
    \usebeamerfont{section title}\insertsection\par
  \end{beamercolorbox}
}
\setbeamertemplate{subsection page}{
  \centering
  \begin{beamercolorbox}[sep=8pt,center]{part title}
    \usebeamerfont{subsection title}\insertsubsection\par
  \end{beamercolorbox}
}
\AtBeginPart{
  \frame{\partpage}
}
\AtBeginSection{
  \ifbibliography
  \else
    \frame{\sectionpage}
  \fi
}
\AtBeginSubsection{
  \frame{\subsectionpage}
}
\usepackage{amsmath,amssymb}
\usepackage{lmodern}
\usepackage{iftex}
\ifPDFTeX
  \usepackage[T1]{fontenc}
  \usepackage[utf8]{inputenc}
  \usepackage{textcomp} % provide euro and other symbols
\else % if luatex or xetex
  \usepackage{unicode-math}
  \defaultfontfeatures{Scale=MatchLowercase}
  \defaultfontfeatures[\rmfamily]{Ligatures=TeX,Scale=1}
\fi
% Use upquote if available, for straight quotes in verbatim environments
\IfFileExists{upquote.sty}{\usepackage{upquote}}{}
\IfFileExists{microtype.sty}{% use microtype if available
  \usepackage[]{microtype}
  \UseMicrotypeSet[protrusion]{basicmath} % disable protrusion for tt fonts
}{}
\makeatletter
\@ifundefined{KOMAClassName}{% if non-KOMA class
  \IfFileExists{parskip.sty}{%
    \usepackage{parskip}
  }{% else
    \setlength{\parindent}{0pt}
    \setlength{\parskip}{6pt plus 2pt minus 1pt}}
}{% if KOMA class
  \KOMAoptions{parskip=half}}
\makeatother
\usepackage{xcolor}
\newif\ifbibliography
\usepackage{graphicx}
\makeatletter
\def\maxwidth{\ifdim\Gin@nat@width>\linewidth\linewidth\else\Gin@nat@width\fi}
\def\maxheight{\ifdim\Gin@nat@height>\textheight\textheight\else\Gin@nat@height\fi}
\makeatother
% Scale images if necessary, so that they will not overflow the page
% margins by default, and it is still possible to overwrite the defaults
% using explicit options in \includegraphics[width, height, ...]{}
\setkeys{Gin}{width=\maxwidth,height=\maxheight,keepaspectratio}
% Set default figure placement to htbp
\makeatletter
\def\fps@figure{htbp}
\makeatother
\setlength{\emergencystretch}{3em} % prevent overfull lines
\providecommand{\tightlist}{%
  \setlength{\itemsep}{0pt}\setlength{\parskip}{0pt}}
\setcounter{secnumdepth}{-\maxdimen} % remove section numbering
% Definição do esquema de cores:
% 1. UFPR - Azul com cinza.
% 2. DEST - Roxo com cinza.
% 3. LEG - Laranjado com cinza.
\def\mycolorscheme{1}

% Caminho para a imagem de fundo com aspecto 16x9.
% \def\pathtobg{config/ufpr-fachada-baixo-1.jpg}
% \def\pathtobg{config/ufpr-fundo.jpg}
% \def\pathtobg{config/ufpr-fundo.jpg}
\def\pathtobg{./config/ufpr-fundo-16x9.jpg}

% \providecommand{\tightlist}{%
%   \setlength{\itemsep}{0pt}\setlength{\parskip}{0pt}}
% ATTENTION: Redefine o comando acima que é definido pelo template.
% \renewcommand{\tightlist}{}
\renewcommand{\tightlist}{%
  \setlength{\itemsep}{0\baselineskip}
  \setlength{\parskip}{0.25\baselineskip}
}

% Logo na capa.
\titlegraphic{
  %\vspace{-1em}
  %\includegraphics[height=1.2cm]{config/dest-texto-2.png}\hspace{1em}
  %\includegraphics[height=1.8cm]{config/dsbd-logo-2x2.png}\hspace{1em}
  \includegraphics[height=1.8cm]{config/ufpr-transparent-600px.png}
}
%-----------------------------------------------------------------------

% Palladio.
% \usepackage[sc]{mathpazo}
% \linespread{1.05}         % Palladio needs more leading (space between lines)
% \usepackage[T1]{fontenc}

% Kurier.
% \usepackage[light, condensed, math]{kurier}
% \usepackage[T1]{fontenc}

% Iwona.
% \usepackage[math, light, condensed]{iwona}

% \usepackage{cmbright}
% \usepackage[charter]{mathdesign}
% \usepackage{palatino}

% Roboto (with Iwona for maths).
% \usepackage[math]{iwona}
% \usepackage[sfdefault, light, condensed]{roboto}

% Source Sans Pro (with Iwona for maths).
% \usepackage[math]{iwona}
% \usepackage[default, light]{sourcesanspro}

% Lato (with Iwona for maths).
% \usepackage[math]{iwona}
% \usepackage[default]{lato}

% Fira Sans (with Iwona for maths).
\usepackage[math, light]{iwona}
\usepackage[sfdefault,light]{FiraSans} %% option 'sfdefault' activates Fira Sans as the default text font
\usepackage[T1]{fontenc}
\renewcommand*\oldstylenums[1]{{\firaoldstyle #1}}

% Font for code. ----------------------------
% \usepackage[scaled=.75]{beramono}
\usepackage{inconsolata}

% ATTENTION: needs complile with xelatex: `$ xelatex file.tex`
% \usepackage{fontspec}
% \setmonofont{M+ 1m}
% \setmonofont{M+ 1mn}
% \setmonofont{M+ 2m}

%-----------------------------------------------------------------------

% \usepackage{lmodern}
\usepackage{amssymb, amsmath}
\usepackage[makeroom]{cancel}
% \usepackage{ifxetex, ifluatex}
\usepackage{fixltx2e} % provides \textsubscript
\usepackage[utf8]{inputenc}
\usepackage[shorthands=off,main=brazil]{babel}
\usepackage{graphicx}
\usepackage{xcolor}
\usepackage{setspace}
\usepackage{comment}
\usepackage{icomma}

%-----------------------------------------------------------------------
% Algumas configurações.

\setlength{\parindent}{0pt}
\setlength{\parskip}{6pt plus 2pt minus 1pt}
\setlength{\emergencystretch}{3em}  % prevent overfull lines
% \providecommand{\tightlist}{%
%   \setlength{\itemsep}{0pt}\setlength{\parskip}{0pt}}
\setcounter{secnumdepth}{0}

% Espaço vertical para o ambiente `quote`.
\let\oldquote\quote
\let\oldendquote\endquote
\renewenvironment{quote}{%
  \vspace{1em}\oldquote}{%
  \oldendquote\vspace{1em}}

%-----------------------------------------------------------------------
% Espaçamento entre items para itemize, enumerate e description.

% % itemize.
% \let\itemopen\itemize
% \let\itemclose\enditemize
% \renewenvironment{itemize}{%
%   \itemopen\addtolength{\itemsep}{0.25\baselineskip}}{\itemclose}
%
% % enumerate.
% \let\enumopen\enumerate
% \let\enumclose\endenumerate
% \renewenvironment{enumerate}{%
%   \enumopen\addtolength{\itemsep}{0.25\baselineskip}}{\enumclose}
%
% % description.
% \let\descopen\description
% \let\descclose\enddescription
% \renewenvironment{description}{%
%   \descopen\addtolength{\itemsep}{0.25\baselineskip}}{\descclose}

%-----------------------------------------------------------------------

% \usepackage[hang]{caption}
\usepackage{caption}
\captionsetup{font=footnotesize,
  labelfont={color=mycolor1, footnotesize},
  labelsep=period}

% \providecommand{\tightlist}{%
%   \setlength{\itemsep}{0pt}\setlength{\parskip}{0pt}}

%-----------------------------------------------------------------------

\usepackage{tikz}

% \def\pathtobg{/home/walmes/Projects/templates/COMMON/ufpr-fundo.jpg}
% \def\pathtobg{/home/walmes/Projects/templates/COMMON/ufpr-fundo-16x9.jpg}
% \def\pathtobg{/home/walmes/Projects/templates/COMMON/ufpr-fachada-dir-1.jpg}
% \def\pathtobg{/home/walmes/Projects/templates/COMMON/ufpr-fachada-esq-1.jpg}
% \def\pathtobg{/home/walmes/Projects/templates/COMMON/ufpr-perto-1.jpg}
% \def\pathtobg{/home/walmes/Projects/templates/COMMON/ufpr-fachada-baixo-1.jpg}

\ifx\pathtobg\undefined
\else
  \usebackgroundtemplate{
    \tikz[overlay, remember picture]
    \node[% opacity=0.3,
          at=(current page.south east),
          anchor=south east,
          inner sep=0pt] {
            \includegraphics[height=\paperheight, width=\paperwidth]{\pathtobg}};
  }
\fi

%-----------------------------------------------------------------------
% Definições de esquema de cores.

\ifx\mycolorscheme\undefined
  % UFPR.
  % http://www.color-hex.com/color-palette/2018
  \definecolor{mycolor1}{HTML}{015c93} % Título.
  \definecolor{mycolor2}{HTML}{363435} % Texto.
  \definecolor{mycolor3}{HTML}{015c93} % Estrutura.
  \definecolor{mycolor4}{HTML}{015c93} % Links.
  \definecolor{mycolor5}{HTML}{CECAC5} % Preenchimentos.
\else
  \if\mycolorscheme1
    % UFPR.
    \definecolor{mycolor1}{HTML}{015c93} % Título.
    \definecolor{mycolor2}{HTML}{363435} % Texto.
    \definecolor{mycolor3}{HTML}{015c93} % Estrutura.
    \definecolor{mycolor4}{HTML}{015c93} % Links.
    \definecolor{mycolor5}{HTML}{CECAC5} % Preenchimentos.
  \fi
  \if\mycolorscheme2
    % DEST.
    \definecolor{mycolor1}{HTML}{2a0e72} % Título.
    \definecolor{mycolor2}{HTML}{202E35} % Texto.
    \definecolor{mycolor3}{HTML}{2a0e72} % Estrutura.
    % \definecolor{mycolor3}{HTML}{8072a3} % Estrutura.
    \definecolor{mycolor4}{HTML}{2a0e72} % Links.
    % \definecolor{mycolor4}{HTML}{bfb9d1} % Links.
    % \definecolor{mycolor5}{HTML}{AEA79F} % Preenchimentos.
    \definecolor{mycolor5}{HTML}{CECAC5} % Preenchimentos.
  \fi
  \if\mycolorscheme3
    % LEG.
    \definecolor{mycolor2}{HTML}{363435} % Texto.
    % \definecolor{mycolor1}{HTML}{ff8000} % Título.
    % \definecolor{mycolor3}{HTML}{ff8000} % Estrutura.
    % \definecolor{mycolor4}{HTML}{ff8000} % Links.
    % \definecolor{mycolor1}{HTML}{E57300} % Título.
    % \definecolor{mycolor3}{HTML}{E57300} % Estrutura.
    % \definecolor{mycolor4}{HTML}{E57300} % Links.
    \definecolor{mycolor1}{HTML}{F67014} % Título.
    \definecolor{mycolor3}{HTML}{F67014} % Estrutura.
    \definecolor{mycolor4}{HTML}{F67014} % Links.
    % \definecolor{mycolor1}{HTML}{FE5C23} % Título.
    % \definecolor{mycolor3}{HTML}{FE5C23} % Estrutura.
    % \definecolor{mycolor4}{HTML}{FE5C23} % Links.
    \definecolor{mycolor5}{HTML}{222222} % Preenchimentos.
    \definecolor{mycolor5}{HTML}{383838} % Preenchimentos.
  \fi
\fi

\hypersetup{
  colorlinks=true,
  linkcolor=mycolor4,
  urlcolor=mycolor1,
  citecolor=mycolor1
}

%-----------------------------------------------------------------------
% ATTENTION: http://www.cpt.univ-mrs.fr/~masson/latex/Beamer-appearance-cheat-sheet.pdf

\usetheme{Boadilla}
\usecolortheme{default}

% \setbeamersize{text margin left=7mm, text margin right=7mm}
% \setbeamertemplate{frametitle}[default][left, leftskip=3mm]
% \addtobeamertemplate{frametitle}{\vspace{0.5em}}{}

\setbeamertemplate{caption}[numbered]
\setbeamertemplate{section in toc}[sections numbered]
\setbeamertemplate{subsection in toc}[subsections numbered]
\setbeamertemplate{sections/subsections in toc}[ball]{}
\setbeamertemplate{sections in toc}[ball]
\setbeamercolor{section number projected}{bg=mycolor1, fg=white}
\setbeamertemplate{blocks}[rounded]
\setbeamertemplate{navigation symbols}{}
\setbeamertemplate{frametitle continuation}{\gdef\beamer@frametitle{}}
% \setbeamertemplate{frametitle}[default][center]
% \setbeamertemplate{footline}[frame number]

\setbeamertemplate{enumerate items}[default]
\setbeamertemplate{itemize items}{\scriptsize\raise1.25pt\hbox{\donotcoloroutermaths$\blacktriangleright$}}

% Blocos.
% \addtobeamertemplate{block begin}{\vskip -\bigskipamount}{}
% \addtobeamertemplate{block end}{}{\vskip -\bigskipamount}
\addtobeamertemplate{block begin}{\vspace{0.5em}}{}
\addtobeamertemplate{block end}{}{\vspace{0.5em}}


% Rodapé.
\setbeamercolor{title in head/foot}{parent=subsection in head/foot}
\setbeamercolor{author in head/foot}{bg=mycolor4, fg=white}
\setbeamercolor{date in head/foot}{parent=subsection in head/foot, fg=mycolor3}

% Cabeçalho.
\setbeamercolor{section in head/foot}{bg=mycolor2, fg=mycolor4}
\setbeamercolor{subsection in head/foot}{bg=mycolor2, fg=white}

\setbeamercolor{title}{fg=mycolor1}       % Título dos slides.
\setbeamercolor{titlelike}{fg=title}
\setbeamercolor{subtitle}{fg=mycolor2}    % Subtítulo.
\setbeamercolor{institute in head/foot}{parent=palette primary} % Instituição.
\setbeamercolor{frametitle}{fg=mycolor1}  % De quadro.
\setbeamercolor{structure}{fg=mycolor3}   % Listas e rodapé.
\setbeamercolor{item projected}{bg=mycolor2}
\setbeamercolor{block title}{bg=mycolor5, fg=mycolor2}
\setbeamercolor{normal text}{fg=mycolor2} % Texto.
\setbeamercolor{caption name}{fg=normal text.fg}
% \setbeamercolor{footlinecolor}{fg=mycolor2, bg=mycolor5}
% \setbeamercolor{section in head/foot}{fg=mycolor2, bg=mycolor5}
\setbeamercolor{author in head/foot}{fg=white, bg=mycolor1}
\setbeamercolor{section in foot}{fg=mycolor4, bg=mycolor5}
\setbeamercolor{date in foot}{fg=mycolor4, bg=mycolor5}
\setbeamercolor{block title}{fg=white, bg=mycolor1}
\setbeamercolor{block body}{fg=black, bg=white!80!gray}
\setbeamercolor{block body}{fg=black, bg=white!80!gray}

% To remove empty brackets of \institution.
\makeatletter
\setbeamertemplate{footline}{
  \leavevmode%
  \hbox{%
    \begin{beamercolorbox}[
      wd=0.3\paperwidth, ht=2.25ex, dp=1ex, right]{author in head/foot}%
      \usebeamerfont{author in head/foot}\insertshortauthor{}\hspace*{1ex}
    \end{beamercolorbox}%
    \begin{beamercolorbox}[
      wd=0.6\paperwidth, ht=2.25ex, dp=1ex, left]{section in foot}%
      \usebeamerfont{title in head/foot}\hspace*{1ex}\insertshorttitle{}
      % \usebeamerfont{title in head/foot}\hspace*{1ex}\insertframetitle{}
    \end{beamercolorbox}%
    \begin{beamercolorbox}[
      wd=0.1\paperwidth, ht=2.25ex, dp=1ex, right]{date in foot}%
      \insertframenumber{}\hspace*{2ex}
    \end{beamercolorbox}
  }%
  \vskip0pt%
}
\makeatother

%-----------------------------------------------------------------------

% \usepackage{hyphenat}
\usepackage{changepage}

% Slide para o título das seções.
\AtBeginSection[]{
  \begin{frame}
    % \vfill
    \vspace{4cm}
    % \centering
    % \begin{beamercolorbox}[sep = 8pt, center, shadow = true, rounded = true]{title}
    \begin{beamercolorbox}{title}
      \begin{columns}
        \column{0.7\linewidth}
        {\LARGE\textbf \insertsectionhead}
      \end{columns}
    \end{beamercolorbox}
    \vfill
  \end{frame}
}

%-----------------------------------------------------------------------
%---- preamble-chunk.tex -----------------------------------------------

% Knitr.

% ATTENTION: this needs `\usepackage{xcolor}'.
\definecolor{color_line}{HTML}{333333}
\definecolor{color_back}{HTML}{DDDDDD}
% \definecolor{color_back}{HTML}{FF0000}

% ATTENTION: usa o fancyvrb.
% https://ctan.math.illinois.edu/macros/latex/contrib/fancyvrb/doc/fancyvrb-doc.pdf
% R input.
\usepackage{tcolorbox}
\ifcsmacro{Highlighting}{
  % Statment if it exists. ------------------
  \DefineVerbatimEnvironment{Highlighting}{Verbatim}{
    % frame=lines,     % Linha superior e inferior.
    % framerule=0.5pt, % Espessura da linha.
    framesep=2ex,    % Distância da linha para o texto.
    % rulecolor=\color{color_line},
    % numbers=right,
    fontsize=\footnotesize, % Tamanho da fonte.
    baselinestretch=0.8,    % Espaçamento entre linhas.
    commandchars=\\\{\}}
  % Margens do ambiente `Shaded'.
  % \fvset{listparameters={\setlength{\topsep}{-1em}}}
  % \renewenvironment{Shaded}{\vspace{-1ex}}{\vspace{-2ex}}
  \renewenvironment{Shaded}{
    \vspace{2pt}
    \begin{tcolorbox}[
      boxrule=0pt,      % Espessura do contorno.
      colframe=gray!10, % Cor do contorno.
      colback=gray!10,  % Cor de fundo da caixa.
      arc=1em,          % Raio para contornos arredondados.
      sharp corners,
      boxsep=0.5em,     % Margem interna.
      left=3pt, right=3pt, top=3pt, bottom=3pt, % Margens internas.
      grow to left by=0mm,
      grow to right by=6pt,
      ]
    }{
    \end{tcolorbox}
    \vspace{-3pt}
    }
  }{
  % Statment if it not exists. --------------
}

% R output e todo `verbatim'.
\makeatletter
\def\verbatim@font{\linespread{0.8}\ttfamily\footnotesize}
%\makeatother

% Cor de fundo e margens do `verbatim'.
\let\oldv\verbatim
\let\oldendv\endverbatim

\def\verbatim{%
  \par\setbox0\vbox\bgroup % Abre grupo.
  %\vspace{-5px}            % Reduz margem superior.
  \oldv                    % Chama abertura do verbatim.
}
\def\endverbatim{%
  \oldendv                 % Chama encerramento do verbatim.
  %\vspace{0cm}           % Controla margem inferior.
  \egroup%\fboxsep5px      % Fecha grupo.
  \noindent{{\usebox0}}\par
}

%-----------------------------------------------------------------------
%---- preamble-commands.tex --------------------------------------------

% Para fazer texto em duas colunas.
\newcommand{\mytwocolumns}[4]{
  % #1: Line width fraction for the left column , e.g. 0.5.
  % #2: Line width fraction for the right column.
  % #3: Content for the left column.
  % #4: Content for the right column.
  \begin{columns}[c]
    \begin{column}{#1\linewidth} %----------- left.
      #3
    \end{column} %--------------------------- left.
    \begin{column}{#2\linewidth} %----------- right.
      #4
    \end{column} %--------------------------- right.
  \end{columns}
}

%-----------------------------------------------------------------------
% Para fazer duas colunas no Rmd.

% Center vertical align.
\def\beginAHalfColumn{\begin{minipage}{0.49\textwidth}}%
\def\beginAlmostHalfColumn{\begin{minipage}{0.45\textwidth}}%
\def\beginAQuarterColumn{\begin{minipage}{0.23\textwidth}}%
\def\beginThreeQuartersColumn{\begin{minipage}{0.72\textwidth}}%
\def\beginAThirdColumn{\begin{minipage}{0.31\textwidth}}%
\def\beginTwoThirdsColumn{\begin{minipage}{0.64\textwidth}}%
\def\endColumns{\end{minipage}}%

% Top vertical align.
\def\beginAHalfColumnT{\begin{minipage}[t]{0.49\textwidth}}%
\def\beginAlmostHalfColumnT{\begin{minipage}[t]{0.45\textwidth}}%
\def\beginAQuarterColumnT{\begin{minipage}[t]{0.23\textwidth}}%
\def\beginThreeQuartersColumnT{\begin{minipage}[t]{0.72\textwidth}}%
\def\beginAThirdColumnT{\begin{minipage}[t]{0.31\textwidth}}%
\def\beginTwoThirdsColumnT{\begin{minipage}[t]{0.64\textwidth}}%

%---------------------------------------------------------------------
% Ambientes para frases como e sem imagem.

\newcommand{\myquote}[3]{
  % #1: caminho para a imagem.
  % #2: a frase/quotation.
  % #3: o autor.
  \begin{center}
    \begin{minipage}[c]{0.19\linewidth}
      \begin{center}
        \includegraphics[height=2.5cm]{#1}
      \end{center}
    \end{minipage}
    \begin{minipage}[c]{0.7\linewidth}
      \begin{flushright}
        \textit{#2}
        \vspace{1ex}

        -- #3
      \end{flushright}
    \end{minipage}
  \end{center}
}

\newcommand{\myphrase}[2]{
  % #1: a frase/quotation.
  % #2: o autor.
  \begin{center}
    \begin{minipage}[c]{0.19\linewidth}
    \end{minipage}
    \begin{minipage}[c]{0.7\linewidth}
      \begin{flushright}
        \textit{#1}
        \vspace{1ex}

        -- #2
      \end{flushright}
    \end{minipage}
  \end{center}
}

%-----------------------------------------------------------------------
% Comandos para texto em destaque.

% \newcommand{\hi}[1]{%
%   \textcolor{ubuntu_orange}{#1}\xspace
% }

\usepackage{xspace}

% URLs com letra miuda.
\newcommand{\myurl}[1]{%
  {\tiny \url{#1}}\xspace
}

% Botões.
\newcommand{\btn}[1]{%
  \beamergotobutton{#1}\xspace
}

% Texto grande centralizado.
\newcommand{\centertitle}[1]{%
  \begin{center}
    {\LARGE \bfseries \hi{#1}}
  \end{center}
}

%-----------------------------------------------------------------------
\ifLuaTeX
  \usepackage{selnolig}  % disable illegal ligatures
\fi
\IfFileExists{bookmark.sty}{\usepackage{bookmark}}{\usepackage{hyperref}}
\IfFileExists{xurl.sty}{\usepackage{xurl}}{} % add URL line breaks if available
\urlstyle{same} % disable monospaced font for URLs
\hypersetup{
  pdfauthor={Prof.~Me. Lineu Alberto Cavazani de Freitas},
  hidelinks,
  pdfcreator={LaTeX via pandoc}}

\title{\textbf{Análise exploratória I}}
\subtitle{Motivação e análise descritiva univariada para variáveis
qualitativas e quantitativas.}
\author{Prof.~Me. Lineu Alberto Cavazani de Freitas}
\date{}
\institute{\textbf{CE003 – Estatística II}\\
\strut \\
Departamento de Estatística\\
Laboratório de Estatística e Geoinformação}

\begin{document}
\frame{\titlepage}

\begin{frame}{Análise exploratória}
\protect\hypertarget{anuxe1lise-exploratuxf3ria}{}
\begin{itemize}
\item
  Parte primordial de qualquer análise estatística é chamada análise
  descritiva ou exploratória.
\item
  Consiste basicamente de tabelas, resumos numéricos e análises gráficas
  das variáveis disponíveis em um conjunto de dados.
\item
  Trata-se de uma etapa de extrema importância e deve preceder qualquer
  análise mais sofisticada.
\item
  As técnicas de análise exploratória visam resumir e apresentar as
  informações de um conjunto de dados brutos.
\end{itemize}
\end{frame}

\begin{frame}{Análise exploratória}
\protect\hypertarget{anuxe1lise-exploratuxf3ria-1}{}
\begin{itemize}
\item
  Tentar compreender um conjunto de dados sem algum método que permita
  resumir as informações é inviável.
\item
  A análise exploratória é a primeira forma de tentarmos enteder o que
  acontece nos nossos dados.
\item
  Compreende também a etapa de consistência dos dados, isto é, verificar
  se os dados coletados são condizentes com a realidade.
\end{itemize}
\end{frame}

\begin{frame}{Análise exploratória}
\protect\hypertarget{anuxe1lise-exploratuxf3ria-2}{}
\begin{itemize}
\item
  O conjunto de técnicas aplicáveis está diretamente associado ao tipo
  das variáveis de interesse (quantitativas x qualitativas) e suas
  ramificações.
\item
  Podemos conduzir análises focadas nas variáveis uma a uma (análises
  univariadas).
\item
  Bem como conduzir análises focadas em avaliar a relação entre as
  variáveis (análises multivariadas).
\end{itemize}
\end{frame}

\begin{frame}{Análise exploratória}
\protect\hypertarget{anuxe1lise-exploratuxf3ria-3}{}
Podemos fazer uso diversas técnicas, tais como

\begin{itemize}
\item
  Medidas de posição central.
\item
  Medidas de posição relativa.
\item
  Medidas de forma.
\item
  Medidas de dispersão.
\item
  Medidas de associação.
\item
  Tabelas de frequência absolutas.
\item
  Tabelas de frequência relativas.
\item
  Tabelas de frequência acumuladas.
\item
  Tabelas para múltiplas variáveis.
\item
  Gráficos (para análise uni e multivariada).
\end{itemize}
\end{frame}

\hypertarget{anuxe1lise-descritiva-univariada-para-variuxe1veis-qualitativas}{%
\section{Análise descritiva univariada para variáveis
qualitativas}\label{anuxe1lise-descritiva-univariada-para-variuxe1veis-qualitativas}}

\begin{frame}{Análise descritiva univariada para variáveis qualitativas}
\protect\hypertarget{anuxe1lise-descritiva-univariada-para-variuxe1veis-qualitativas-1}{}
\begin{itemize}
\item
  Uma variável qualitativa representa um atributo que pode ser expresso
  por meio de rótulos ou categorias.
\item
  Podem ser classificadas em nominais (sem ordenação natural entre os
  níveis) ou ordinais (com ordenação natural entre os níveis).
\item
  As categorias também são chamadas de classes ou níveis.
\item
  Na análise descritiva de uma variável qualitativa estamos interessados
  em avaliar as frequências das classes.
\end{itemize}
\end{frame}

\begin{frame}{Tipos de frequência}
\protect\hypertarget{tipos-de-frequuxeancia}{}
\begin{itemize}
\item
  Frequência absoluta: número de observações no conjunto de dados que
  pertence a uma determinada classe.
\item
  Frequência relativa: frequência de classe dividida pelo número total
  de observações no conjunto de dados.

  \begin{itemize}
  \tightlist
  \item
    Pode ser apresentada em forma de percentual, quando multiplicada por
    100.
  \end{itemize}
\item
  Frequência acumulada: frequência absoluta ou relativa acumulada
  conforme disposição das classes.

  \begin{itemize}
  \tightlist
  \item
    Não faz muito sentido para variáveis qualitativas nominais.
  \end{itemize}
\end{itemize}
\end{frame}

\begin{frame}{Exemplos}
\protect\hypertarget{exemplos}{}
\end{frame}

\begin{frame}{Tabelas de frequência para uma variável qualitativa}
\protect\hypertarget{tabelas-de-frequuxeancia-para-uma-variuxe1vel-qualitativa}{}
\begin{itemize}
\item
  Utlizando apenas os dados brutos é difícil responder questões de
  interesse.
\item
  Para reduzir os dados originais de forma que fique mais claro o
  entendimento dos mesmos são utilizadas as tabelas de frequência.
\item
  No caso de variáveis qualitativas consiste em listar os possíveis
  níveis da variável e fazer a contagem de quantas vezes cada nível
  aparece nos dados brutos.
\end{itemize}
\end{frame}

\begin{frame}{Tabelas de frequência para uma variável qualitativa}
\protect\hypertarget{tabelas-de-frequuxeancia-para-uma-variuxe1vel-qualitativa-1}{}
\begin{itemize}
\item
  Cada linha da tabela diz respeito a um nível da variável categórica.
\item
  As colunas podem apresentar diferentes tipos de frequência (absoluta,
  relativa, acumulada).
\item
  Alguns cuidados para a apresentação dos resultados dizem respeito ao
  tipo de variável categórica em questão: nominal ou ordinal.
\end{itemize}
\end{frame}

\begin{frame}{Tabelas de frequência para uma variável qualitativa}
\protect\hypertarget{tabelas-de-frequuxeancia-para-uma-variuxe1vel-qualitativa-2}{}
\begin{itemize}
\item
  Os níveis de variáveis nominais não apresentam uma ordenação natural,
  portanto, na apresentação dos resultados pode ser interessante ordenar
  os níveis por frequência ou por ordem alfabética.
\item
  Esta estratégia não é recomendada para variáveis ordinais, pois estas
  apresentam uma ordenação natural e esta ordenação deve ser
  preferencialmente mantida na exposição dos resultados.
\end{itemize}
\end{frame}

\begin{frame}{Tabelas de frequência para uma variável qualitativa}
\protect\hypertarget{tabelas-de-frequuxeancia-para-uma-variuxe1vel-qualitativa-3}{}
EXEMPLO QUALITATIVA NOMINAL
\end{frame}

\begin{frame}{Tabelas de frequência para uma variável qualitativa}
\protect\hypertarget{tabelas-de-frequuxeancia-para-uma-variuxe1vel-qualitativa-4}{}
EXEMPLO QUALITATIVA ORDINAL
\end{frame}

\begin{frame}{Gráficos para representação de frequências de uma variável
qualitativa}
\protect\hypertarget{gruxe1ficos-para-representauxe7uxe3o-de-frequuxeancias-de-uma-variuxe1vel-qualitativa}{}
\begin{itemize}
\item
  A representação por meio de tabelas é útil mas nem sempre eficiente.
\item
  Em diversos casos pode ser mais conveniente utilizar um gráfico.
\item
  ``Uma imagem vale mais que mil palavras''.
\item
  Os cuidados com a ordenação dos níveis de acordo com o tipo da
  variável se mantém.
\end{itemize}
\end{frame}

\begin{frame}{Gráficos para representação de frequências de uma variável
qualitativa}
\protect\hypertarget{gruxe1ficos-para-representauxe7uxe3o-de-frequuxeancias-de-uma-variuxe1vel-qualitativa-1}{}
Algumas possibilidades são:

\begin{itemize}
\tightlist
\item
  Gráfico de barras verticais.
\item
  Gráfico de barras horizontais.
\item
  Gráfico de barras empilhadas.
\item
  Gráfico de setores.
\item
  Gráfico de rosca.
\end{itemize}
\end{frame}

\begin{frame}{Gráfico de barras verticais ou horizontais}
\protect\hypertarget{gruxe1fico-de-barras-verticais-ou-horizontais}{}
\begin{itemize}
\tightlist
\item
  Utiliza os possíveis níveis das variáveis em um eixo.
\item
  As frequências ou porcentagens ficam no outro eixo.
\item
  O tamanho da barra correspondente à frequência.
\end{itemize}
\end{frame}

\begin{frame}{Gráfico de barras verticais}
\protect\hypertarget{gruxe1fico-de-barras-verticais}{}
EXEMPLO
\end{frame}

\begin{frame}{Gráfico de barras horizontais}
\protect\hypertarget{gruxe1fico-de-barras-horizontais}{}
EXEMPLO
\end{frame}

\begin{frame}{Gráfico de barras empilhadas}
\protect\hypertarget{gruxe1fico-de-barras-empilhadas}{}
\begin{itemize}
\tightlist
\item
  No caso de uma barra empilhada representa-se a frequência relativa ou
  o percentual.
\item
  Existe uma única barra que representa 100\%.
\item
  Esta barra é dividida de acordo com a contribuição relativa de cada
  nível da variável.
\end{itemize}
\end{frame}

\begin{frame}{Gráfico de barras empilhadas}
\protect\hypertarget{gruxe1fico-de-barras-empilhadas-1}{}
EXEMPLO
\end{frame}

\begin{frame}{Gráfico de setores}
\protect\hypertarget{gruxe1fico-de-setores}{}
\begin{itemize}
\tightlist
\item
  Consiste em repartir um círculo em setores de tamanhos proporcionais
  às frequências relativas ou às porcentagens de cada valor.
\item
  Podem ser usados para representar variáveis com poucos níveis.
\item
  Para variáveis com muitos níveis, o gráfico tende a ficar visualmente
  poluído e pouco informativo.
\item
  Além disso, o cérebro humano tem dificuldade em relacionar frequências
  com áreas relativas.
\item
  Apesar de muito usado e preferido em diversas áreas, deve ser evitado.
\item
  Uma variação do gráfico de setores é o gráfico de rosca.
\end{itemize}
\end{frame}

\begin{frame}{Gráfico de setores}
\protect\hypertarget{gruxe1fico-de-setores-1}{}
EXEMPLO
\end{frame}

\hypertarget{anuxe1lise-descritiva-univariada-para-variuxe1veis-quantitativas}{%
\section{Análise descritiva univariada para variáveis
quantitativas}\label{anuxe1lise-descritiva-univariada-para-variuxe1veis-quantitativas}}

\begin{frame}{Análise descritiva univariada para variáveis
quantitativas}
\protect\hypertarget{anuxe1lise-descritiva-univariada-para-variuxe1veis-quantitativas-1}{}
\begin{itemize}
\item
  Uma variável quantitativa é uma característica que pode ser
  representada numericamente.
\item
  Podem ser classificadas em discretas (finitos valores em um dado
  intervalo) ou contínuas (infinitos valores em um dado intervalo).
\item
  Quando estamos lidando com variáveis quantitativas discretas com
  poucos possíveis valores, as técnicas apresentadas para variáveis
  qualitativas se aplicam.
\end{itemize}
\end{frame}

\begin{frame}{Tabelas de frequência}
\protect\hypertarget{tabelas-de-frequuxeancia}{}
EXEMPLO TABELA PARA QUANTITATIVA DISCRETA COM POUCOS NÍVEIS
\end{frame}

\begin{frame}{Gráficos de frequência}
\protect\hypertarget{gruxe1ficos-de-frequuxeancia}{}
EXEMPLOS GRÁFICOS PARA QUANTITATIVA DISCRETA COM POUCOS NIVEIS
\end{frame}

\begin{frame}{Análise descritiva univariada para variáveis
quantitativas}
\protect\hypertarget{anuxe1lise-descritiva-univariada-para-variuxe1veis-quantitativas-2}{}
\begin{itemize}
\item
  Para variáveis quantitativas contínuas ou discretas com muitos
  possíveis valores, precisamos de técnicas específicas.
\item
  Uma estratégia comum é o agrupamento em faixas de valores, e avaliação
  das frequências nestas faixas.
\item
  Cuidados devem ser tomados quanto às notações e tipos de faixas
  (aberto e fechado à esquerda ou direita).
\item
  Podem ser usadas tabelas de frequências absolutas, relativas e
  acumuladas para as faixas de valores.
\item
  Utilizando a razão entre frequência relativa e a amplitude das faixas
  de valores, geramos a densidade.
\end{itemize}
\end{frame}

\begin{frame}{Análise descritiva univariada para variáveis
quantitativas}
\protect\hypertarget{anuxe1lise-descritiva-univariada-para-variuxe1veis-quantitativas-3}{}
\begin{itemize}
\item
  Como agrupar em classes?
\item
  Qual o tamanho ideal das faixas de valores?
\item
  Classes definidas com a mesma amplitude é o procedimento mais usual.
\item
  Existem procedimentos que podem ser usados para obter a amplitude,
  como Sturges.
\end{itemize}
\end{frame}

\begin{frame}{Tabelas de frequência para uma variável quantitativa}
\protect\hypertarget{tabelas-de-frequuxeancia-para-uma-variuxe1vel-quantitativa}{}
EXEMPLO
\end{frame}

\begin{frame}{Gráficos para representação de frequências de uma variável
quantitativa}
\protect\hypertarget{gruxe1ficos-para-representauxe7uxe3o-de-frequuxeancias-de-uma-variuxe1vel-quantitativa}{}
\begin{itemize}
\tightlist
\item
  Assim como no caso de variáveis qualitativas ou quantitativas
  discretas com poucos possíveis valores, a representação por meio de
  gráficos pode ser bastante benéfica para análise de variáveis
  quantitativas.
\end{itemize}

Algumas possibilidades são

\begin{itemize}
\tightlist
\item
  Histograma da frequência absoluta (afetado pelo número de classes).
\item
  Histograma com amplitude de classe variável.
\item
  Histograma da densidade.
\item
  Histograma da densidade para amplitude de classe variável.
\item
  Gráfico de frequências acumuladas.
\item
  Gráfico de densidade empírica.
\item
  Box-plot
\end{itemize}
\end{frame}

\begin{frame}{Histograma}
\protect\hypertarget{histograma}{}
\begin{itemize}
\tightlist
\item
  Consiste em retângulos contíguos de base dada pelas faixas de valores
  definindas para uma variável.
\item
  A área igual é igual à frequência da rescpectiva faixa.
\item
  Em uma possível representação a altura pode representar a frequência
  absoluta na faixa de valores.
\item
  Outra possibilidade é a altura de cada retângulo representar o
  quociente da área pela amplitude da faixa: a densidade.
\end{itemize}
\end{frame}

\begin{frame}{Histograma}
\protect\hypertarget{histograma-1}{}
EXEMPLO
\end{frame}

\begin{frame}{Gráfico de densidade empírica}
\protect\hypertarget{gruxe1fico-de-densidade-empuxedrica}{}
\end{frame}

\begin{frame}{Assimetria}
\protect\hypertarget{assimetria}{}
\end{frame}

\begin{frame}{Gráfico de frequências acumuladas}
\protect\hypertarget{gruxe1fico-de-frequuxeancias-acumuladas}{}
\end{frame}

\begin{frame}{Box-plot}
\protect\hypertarget{box-plot}{}
\begin{itemize}
\item
  Outra importante visualização é o box-plot.
\item
  É possível analisar a distribuição dos dados, aspectos quanto a
  posição, variabilidade, assimetria e também a presença de valores
  atípicos.
\item
  Retomaremos o box-plot após estudar quartis, em medidas descritivas.
\end{itemize}

EXEMPLO BOX-PLOT
\end{frame}

\end{document}
